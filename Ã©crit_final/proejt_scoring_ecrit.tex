\documentclass[11pt, oneside]{article}   	% use "amsart" instead of "article" for AMSLaTeX format
\usepackage{geometry}                	%permet de régler des marges 	

\usepackage{graphicx}		% package pour travailler sur des images
\geometry{top=1.22cm, bottom=2.2cm, left=1.7cm , right=1.7cm}     %Réxgler les marges
\usepackage{fancyhdr}   % en têtes et pied de pages personnalisés 
\pagestyle{fancy}             %SetFonts

\renewcommand{\headrulewidth}{1pt}
\fancyhead[L]{\leftmark}
\fancyhead[R]{\rightmark}




\usepackage{amsmath,amssymb,amsthm}  %packages maths    amsthm: théorèmes
\usepackage[fleqn]{mathtools}
\usepackage[frenchb]{babel}  %accents    écrire doc en français 
\usepackage[francais]{babel}

%\usepackage[thmmarks,amsmath]{ntheorem}
\usepackage{bbm}
\usepackage{dsfont}
\usepackage{thmtools}   % theoreme 
\usepackage[unq]{unique}
\usepackage{mathrsfs}


\usepackage[T1]{fontenc}
\usepackage{float} %gestion positionnement de l'image sur la feuille 
\usepackage{graphicx}     %gestion insertion d'images \usepackage{setspace}
\usepackage{abstract}
\usepackage{subfig}
\newcommand{\HRule}{\rule{\linewidth}{0.5mm}}
\usepackage{caption}
\usepackage{lipsum}


%SetFonts
\def\restriction#1#2{\mathchoice
              {\setbox1\hbox{${\displaystyle #1}_{\scriptstyle #2}$}
              \restrictionaux{#1}{#2}}
              {\setbox1\hbox{${\textstyle #1}_{\scriptstyle #2}$}
              \restrictionaux{#1}{#2}}
              {\setbox1\hbox{${\scriptstyle #1}_{\scriptscriptstyle #2}$}
              \restrictionaux{#1}{#2}}
              {\setbox1\hbox{${\scriptscriptstyle #1}_{\scriptscriptstyle #2}$}
              \restrictionaux{#1}{#2}}}
\def\restrictionaux#1#2{{#1\,\smash{\vrule height .8\ht1 depth .85\dp1}}_{\,#2}} 

%\bibliographystyle{plain}


\newtheorem{theo}{Théorème}[section]
% \theoremstyle{definition} pour enlever italique 
\newtheorem{definition}{Définition}[section]
\newtheorem{prop}{Proposition}[section]
\newtheorem{pro}{Propriété}[section]
\newtheorem{cor}{Corollaire}[section]
\newtheorem{lem}{Lemme}[section]							% TeX will automatically convert eps --> pdf in pdflatex	

\declaretheorem[title=Théorème, style=break, shaded]{theor}  

\DeclareMathOperator{\sinc}{sinc}  %pour le sinus cardinal
\DeclareMathOperator\supp{supp}


\usepackage{amssymb}


%%%%%%%%%%%%%%%%%%%%  RACCOURCIS %%%%%%%%%%%%%%%%%%%%
\newcommand{\R}{\mathbb{R}}
\newcommand{\C}{\mathbb{C}}
\newcommand{\Z}{\mathbb{Z}}
\newcommand{\N}{\mathbb{N}}
\newcommand{\PP}{\mathbb{P}}
\newcommand{\EE}{\mathbb{E}}
\renewcommand{\inf}{\infty}
\newcommand{\ra}{\rightarrow}
\newcommand{\lra}{\longrightarrow}
\newcommand{\Lra}{\Leftrightarrow}
%SetFonts


%ecriture algo 
\usepackage{listings}
\usepackage{xcolor}
\usepackage{color}
 \lstset{style=mystyle}



% \usepackage{listings} % to inlude R code with \begin{lstlisting}[language=R] etc.
\usepackage[usenames,dvipsnames]{xcolor}
\definecolor{backgroundCol}{rgb}{.97, .97, .97}   %couleur définie pour le fond
\definecolor{commentstyleCol}{rgb}{0.678,0.584,0.686}  %couleur définie pour les commentaires
\definecolor{keywordstyleCol}{rgb}{0.737,0.353,0.396}  %couleur définie pour les mots clés
\definecolor{stringstyleCol}{rgb}{0.192,0.494,0.8} %couleur définie pour les chaines de caracteres
\definecolor{NumCol}{rgb}{0.686,0.059,0.569} 
\definecolor{basicstyleCol}{rgb}{0.345, 0.345, 0.345} % couleur définie de base      
\lstset{ 
  language=R,                     % language du code 
  basicstyle=\small  \ttfamily \color{basicstyleCol}, % taille de la police 
  numbers=left,                   % pour mettre les numeros de lignes, à commenter si t'utilises pas
 numberstyle=\tiny\color{green},  % style utilisé pour les numeros des lignes 
  stepnumber=2,                   % ecart entre chaque num: 1: numerote toutes les lignes 
  numbersep=5pt,                  % how far the line-numbers are from the code
  backgroundcolor=\color{backgroundCol},  % choisir le style de fond utilise \usepackage{color}
  showspaces=false,               % pour montrer les espaces
  showstringspaces=false,         % montrer les espaces des chaines de caracteres
  showtabs=false,                 % show tabs within strings adding particular underscores
  %frame=single,                   % ajouter un cadre autour du code 
  %rulecolor=\color{white},        % if not set, the frame-color may be changed on line-breaks within not-black text (e.g. commens (green here))
  tabsize=2,                      % sets default tabsize to 2 spaces
  captionpos=b,                   % sets the caption-position to bottom
  breaklines=true,                % sets automatic line breaking
  breakatwhitespace=false,        % sets if automatic breaks should only happen at whitespace
  keywordstyle=\color{keywordstyleCol},      % les couleurs/style des mots clés
  commentstyle=\color{commentstyleCol},   % style des commentaires (couleurs)
  stringstyle=\color{stringstyleCol},      % style des chaines de caracteres
  literate=%
   *{0}{{{\color{NumCol}0}}}1
    {1}{{{\color{NumCol}1}}}1
    {2}{{{\color{NumCol}2}}}1
    {3}{{{\color{NumCol}3}}}1
    {4}{{{\color{NumCol}4}}}1
    {5}{{{\color{NumCol}5}}}1
    {6}{{{\color{NumCol}6}}}1
    {7}{{{\color{NumCol}7}}}1
    {8}{{{\color{NumCol}8}}}1
    {9}{{{\color{NumCol}9}}}1
} 



\title{Projet}
\author{Nadia GHERNAOUT \\ Philippine RENAUDIN}
\date{}							% Activate to display a given date or no date

\begin{document}
\maketitle
\section{Introduction}

%rajouter histoire du scoring 

Le but est d'affecter un \textit{score} à un individu par rapport à son comportement en comparaison au comportement général d'un panel. Ce score sert ensuite de critère de classification : l'individu sera affecté à un groupe en fonction de celui-ci. 
\paragraph{Exemples d'applications:}
\begin{enumerate}
\item risque médical: prédire si le patient présente des prédispositions à développer une certaine maladie
\item risque financier: prédire la bonne ou mauvaise santé d'une entreprise 
\end{enumerate}
Un score peut également être définit comme un outil statistique ou probabiliste de détection de risque. 
Plusieurs méthodes permettent de construire un score :
\begin{itemize}
\item La régression logistique
\item L'analyse factorielle discriminante 
\end{itemize}

\section{Régression logistique}
\subsection{Introduction}
La Régression logistique consiste à expliquer une variable $Y$ (variable cible), par une ou plusieurs variables explicatives.  \\

%mettre histoire de esperance conditionnelle 

Dans ce projet on se concentre au cas où la variable à expliquer est binaire. \\
Quand le nombre de modalités de la variable à expliquer est supérieur à 2 on parle de régression logistique \textit{polytomique} (scrutin à plus de deux candidats, degrés de satisfaction pour un produit, mention à un examen...) 
Les avantages de cette méthode sont qu'il n'y a pas besoin d'hypothèses de multinormalité.
Dans la régression logistique 

\[  \mathbb{P} (G1|x) = \PP(Y=1|X=x) = \pi(x) = \dfrac{e^{\beta_0 + \beta_1x_1 + \cdots + \beta_px_p}}{1 + e^{\beta_0 + \beta_1 x_1 + \cdots + \beta_px_p}} \]

On note: $ \beta = \begin{pmatrix}
\beta_1 \\ \beta_2 \\ \vdots \\ \beta_p \end{pmatrix}$ et $x = \begin{pmatrix}
x_1 \\ x_2 \\ \vdots \\ x_p
\end{pmatrix}

\[ \pi(x) = \dfrac{e^{\beta_0 + \beta x }}{1 + e^{\beta_0 + \beta x}} \]

La fonction $ \pi(x)$ est appelée \textit{fonction logistique}. Sa représentation graphique est une sigmoïde en fonction des modalités de $x$. La fonction $\pi(x)$ est comprise dans $[0,1]$, elle convient donc à une probabilité et donne souvent une bonne représentation des phénomènes.

On cherche à écrire l'espérance conditionnelle de la variable à expliquer $Y$ comme combinaison linéaire de variables à expliquer $X$.
On veut modéliser l'espérance conditionnelle $\EE [Y |X=x]$.\\
On cherche la valeur moyenne de $Y$ pour toute valeurs de $X$.

\[ logit(\pi(x)) =\ln \left( \dfrac{\pi(x)}{1-\pi(x) \right) = \beta_0 + \beta_1 + \cdots + \beta_px_p }\]
Utilisée avec la fonction de logarithme néperien, logit est la réciproqie de $f(x) = \dfrac{1}{1 + e^{-x}}$ qui est utilisée pour linéariser les fonctions logistiques. 



\begin{lstlisting}[langage = R]
#librairies
library(ROCR)
library(ggplot2)
library(GGally)
library(forestmodel)
library(effects)
library(ggeffects)
library(boot)
library(corrplot)
library(dplyr)
library(devtools)
library(caTools)  

#fichier
diabete = read.table('/Users/Nadia/Documents/Maths/DATA_SCIENCES/PROJET_SCORING/diabetes.csv', header = TRUE, sep= ',')

attach(diabete)

#Colonnes supp
Breaks_age = c(min(Age),24,29,41, max(Age)
Breaks_preg = c(0,1,3,6, max(Pregnancies))

diabete$Age_classe = cut(Age, breaks = Breaks_age, include.lowest = TRUE)
diabete$Pregnancies_classe = cut(Pregnancies, breaks = Breaks_preg, include.lowest = TRUE)
diabete$sqrtInsulin = (sqrt(Insulin)-1)*2
  #boxcox
# diabete$logInsulin = log(Insulin) pose probleme, les valeurs nulles

diabete
summary(diabete)

diabete[,9] <- factor(diabete[,9])
str(diabete)    # 0: pas de diabete, 1: diabete 


db_cor <- round(cor(diabete[1:8]),1)

corrplot(db_cor)   #pas beaucoup de correlation entre les variables 
N = 1000
mod_ret = rep(0,N)

scores_A = rep(0,10)


for (k in 1:N){
  sample = sample.split(Outcome, SplitRatio = 0.8)
  train = subset(diabete, sample == TRUE)
  test = subset(diabete, sample == FALSE) 
  
  modele_1 = glm(Outcome ~ . - Age_classe - sqrtInsulin - Pregnancies_classe , data = train, family = binomial(link = "logit"))   
  modele_2 = glm(Outcome ~ . -Insulin - Age_classe - Pregnancies_classe , data = train, family=binomial(link="logit")) 
  modele_3 = glm(Outcome ~ Pregnancies + Glucose + BMI + DiabetesPedigreeFunction + BloodPressure + Age, data = train, family = binomial(link = "logit"))
  modele_4 = glm(Outcome ~  Glucose + BMI + DiabetesPedigreeFunction , data = train, family = binomial(link = "logit"))
  modele_5 = glm(Outcome ~  Glucose + BMI + DiabetesPedigreeFunction + BloodPressure + Pregnancies, data = train, family = binomial(link = "logit")) 
  modele_6 = glm(Outcome ~  Glucose + BMI + DiabetesPedigreeFunction + BloodPressure + Age, data = train, family = binomial(link = "logit")) 
  modele_7 = glm(Outcome ~  . - Age - sqrtInsulin - Pregnancies_classe , data = train, family = binomial(link = "logit"))  #modele complet avec les classes d'âge
  modele_8 = glm(Outcome ~  Glucose + BMI + DiabetesPedigreeFunction + BloodPressure + Age_classe, data = train, family = binomial(link = "logit")) 
  modele_9 = glm(Outcome ~  Glucose + BMI + DiabetesPedigreeFunction + Age_classe, data = train, family = binomial(link = "logit")) 
  modele_10 = glm(Outcome ~  Glucose + BMI + DiabetesPedigreeFunction + BloodPressure + Pregnancies_classe, data = train, family = binomial(link = "logit"))  
  
  liste_modeles = list(modele_1, modele_2, modele_3, modele_4, modele_5, modele_6, modele_7, modele_8, modele_9, modele_10)
  n = length(liste_modeles)
  erreur = 1
  j = 0           #numéro du modèle retenu
  A = matrix(0, nrow = 2, ncol=n)
  A[1,]= 1:n
  
  for (i in 1:n){
    outcome.pred = predict(liste_modeles[[i]], newdata=test, type="response")  #rend un score
    erreur_pred = prop.table(table(outcome.pred>0.3, test$Outcome))[3]   #rend le taux de faux negatifs
    A[2,i] = erreur_pred
#    if (erreur_pred<erreur){  #on cherche à minimiser les faux négatifs 
    #  erreur<-erreur_pred  
    #  j = as.integer(i) #modele i retenu
     # } 
  }
  A_tri = A[,order(A[2,], decreasing = FALSE)]
  for (i in 1:n){ 
    scores_A[A_tri[1,i]] = scores_A[A_tri[1,i]] + (11-i)}
 #mod_ret[k]=j
}
scores_A
#attribuer des scores: matrice deux lignes, une avec les modeles une avec les scores 

res = as.data.frame(prop.table(table(mod_ret))); res
max_occ = which.max(res$Freq)
cat("Le modèle à retenir est le modèle", max_occ)

"Modèle final"

Reg_fin = glm(Outcome ~  Glucose + BMI + DiabetesPedigreeFunction + BloodPressure + Age_classe, data = diabete, family = binomial(link = "logit"))
outcome_pred = rep(0,dim(diabete)[1])     #vecteur de 0  

for (i in 1: dim(diabete)[1]){
  if (Reg_fin$fitted.values[i] >=0.3){
    outcome_pred[i] = 1
  }   #ce qu'on a prédit 
}
score_pred = filter(data.frame(Reg_fin$fitted.values, outcome_pred), outcome_pred == "1")  #score predit, ceuw qu'on a predit comme étant malades
score = filter(data.frame(Reg_fin$fitted.values, diabete$Outcome), diabete$Outcome == "1")  #vrai score, ceux qui sont 
score_1 = filter(data.frame(Reg_fin$fitted.values, diabete$Outcome), diabete$Outcome == "1")  #vrai score, ceux qui sont 
score_0 = filter(data.frame(Reg_fin$fitted.values, diabete$Outcome), diabete$Outcome == "0")  #vrai score, ceux qui sont 


hgA = hist(score_pred$Reg_fin.fitted.values, breaks=10, plot=F)   
hgB1 = hist(score_1$Reg_fin.fitted.values, breaks=20, plot=F)
hgB2 = hist(score_0$Reg_fin.fitted.values, breaks=20, plot=F)


col_1 = rgb(1,0,0,0.5)
col_2 = rgb(0,0,1,0.5)

plot(hgB1, col= col_1, freq=FALSE, xlim=c(0,1), ylim= c(0,3))
plot(hgB2, col=col_2, freq=FALSE, xlim=c(0,1), add=T)
lines(density(score_0$Reg_fin.fitted.values), lwd=1.5)
lines(density(score_1$Reg_fin.fitted.values), lwd=1.5)

\end{lstlisting}

\section{Scoring}
\end{document}